\documentclass{article}\usepackage{helvet}\renewcommand{\familydefault}{\sfdefault}\usepackage[letterpaper,top=2cm,bottom=2cm,left=3cm,right=3cm,marginparwidth=1.75cm]{geometry}\usepackage[colorlinks=true,allcolors=red]{hyperref}\usepackage{enumitem}\usepackage{tabularx}\usepackage[T1]{fontenc}\setlist[enumerate,1]{label=\arabic*., left=0pt}\setlist[enumerate,2]{label=\alph*., left=0.5em}\setlist[enumerate,3]{label=\alph*\alph*., left=1em}\setlist[enumerate,4]{label=-, left=1.5em}\setlist{nosep}\setlength{\parindent}{0pt}
\begin{document}
	\hypertarget{label}{
	\begin{center}
		\textbf{§ 1\\Studienaufbau}
	\end{center}
}
	\begin{enumerate}[start=1,label=(\arabic*)]
		\item{Das Masterstudium hat eine Regelstudienzeit von vier Semestern. Das Studienvolumen umfasst 120 LP und etwa 80 Semesterwochenstunden.}
	\end{enumerate}

	\begin{enumerate}[start=2,label=(\arabic*)]
		\item{Es müssen Wahlpflichtmodule im Umfang von insgesamt 90 LP erfolgreich abgeschlossen werden.
		\begin{enumerate}[label=\arabic*.]
			\item{Informatikmodule im Gesamtumfang von 75 bis 80 LP aus den folgenden Bereichen:
			\begin{enumerate}[label=\alph*)]
				\item{Wahlpflichtmodule Informatik (Msc-Inf-WP): Es müssen mindestens 12 LP aus dem Bereich Theoretische Informatik (MSc-Inf-Theo) erbracht werden. Über die Zuordnung von Modulen zu diesem Bereich entscheidet der Prüfungsausschuss nach Rücksprache mit den Dozentinnen und Dozenten des Bereichs und macht diese in geeigneter Weise bekannt.}
				\item{\hypertarget{numb}{}Masterseminare zur Informatik (Msc-Inf-Sem) im Umfang von 5 LP: Ziel eines Masterseminars ist eine eigenständige Auseinandersetzung mit wissenschaftlichen Themen der Informatik, dem Schreiben wissenschaftlicher Texte und dem Präsentieren wissenschaftlicher Ergebnisse.}
				\item{Masterprojekte zur Informatik (Msc-Inf-Proj) im Umfang von 10 LP: Ziel eines Masterprojekts ist die intensive, praktische Auseinandersetzung mit einem aktuellen Thema der Informatik. Inhalt sollen insbesondere die Bereiche Problemanalyse, Spezifikation und Implementierung sein. Das Masterprojekt soll in der Regel als Gruppenarbeit erfolgen, so dass neben den fachlichen Inhalten auch Aspekte der Gruppen- und Projektarbeit erlernt werden. Die Ergebnisse des Masterprojekts werden im Rahmen eines Vortrags präsentiert.}
				\item{Projektgruppen (Msc-Winf-PrGrp) im Umfang von 15 bis 20 LP: Die Projektgruppe verfolgt dieselben Ziele wie ein Masterprojekt in einem größeren Rahmen.}
				\item{\hypertarget{nume}{}Forschungsprojekte (MSc-WInf-FoPro, Mitarbeit in einer Arbeitsgruppe), jeweils im Umfang von bis zu 10 LP.}
			\end{enumerate}
Es müssen die folgenden Bedingungen bei den Modulen gemäß den Buchstaben \hyperlink{numb}{b} bis \hyperlink{nume}{e} eingehalten werden:
			\begin{enumerate}[label=\alph*)]
				\item{Ein oder zwei Masterseminare.}
				\item{Ein oder zwei Masterprojekte oder eine Projektgruppe.}
				\item{Höchstens ein Forschungsprojekt.}
				\item{Der Gesamtumfang der Module gemäß den Buchstaben \hyperlink{numb}{b} bis \hyperlink{nume}{e} ist höchstens 25 LP.}
			\end{enumerate}
}
			\item{\hypertarget{num2}{}Außerfachlicher Wahlbereich im Umfang von 10 bis 15 LP: In diesem Bereich können Studierende Module aus dem Angebot der Christian-Albrechts-Universität zu Kiel wählen, welche nicht auch in einem anderen Bereich dieses Studiengangs belegt werden können. Sprachkurse, welche nicht über das Niveau der gymnasialen Oberstufe hinausgehen, oder die Muttersprache betreffen, sowie Module mit informatischem beziehungsweise wirtschaftsinformatischem Inhalt, gehören nicht zu diesem Bereich. Bei der Wahl außerfachlicher Module müssen die Kapazitätsbeschränkungen anderer Fächer gemäß § 9 Absatz 3 PVO beachtet werden.}
			\item{Neben der Wahl von Modulen gemäß Nummer \hyperlink{num2}{2} ist im außerfachlichen Wahlbereich auch die Wahl eines koordinierten Nebenfachs möglich. Das Nebenfach kann sowohl konsekutiven Charakter haben und das gleiche Nebenfach aus dem Bachelorstudiengang fortsetzen als auch ein neues einführendes Nebenfach sein. Die möglichen Nebenfächer mit den zu absolvierenden Modulen werden zum Studienbeginn in geeigneter Weise durch das Institut für Informatik bekannt gemacht. Dabei kann es in einzelnen Nebenfächern erforderlich sein, mehr als 10 LP zu erreichen. Weitere Nebenfächer können in Absprache mit den beteiligten Fächern und dem Prüfungsausschuss Informatik, bestimmt werden. Das Nebenfach wird auf dem Zeugnis ausgewiesen. \footnote{\hypertarget{fn1}{}\label{fn1}Diese Fußnote referenziert eine andere Fußnote.}}
		\end{enumerate}
}
	\end{enumerate}

	\begin{enumerate}[start=3,label=(\arabic*)]
		\item{Im Rahmen des Masterstudiums ist eine Masterarbeit im Umfang von 30 LP anzufertigen. Näheres regelt § {\Large ??}. Die Masterarbeit erfolgt in der Regel als Abschluss des Masterstudiums.}
	\end{enumerate}

	\begin{enumerate}[start=4,label=(\arabic*)]
		\item{Aktualisierungen der Wahlpflichtbereiche nimmt der Prüfungsausschuss vor; vor Einführung eines neuen Moduls werden die durchführenden Lehrpersonen und die Studiengangskoordinatorin oder der Studiengangskoordinator gehört. \footnote{\hypertarget{fn2}{}\label{fn2}Andere Fußnote.}}
	\end{enumerate}

	\begin{enumerate}[start=5,label=(\arabic*)]
		\item{Bereits für einen Bachelorabschluss verwendete Wahlpflichtmodule können nicht erneut eingebracht werden}
	\end{enumerate}

\end{document}
